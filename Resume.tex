\documentclass[a4paper, 8pt]{article}
\usepackage[a4paper, left=20mm, right=10mm, top=20mm, bottom=20mm]{geometry}
\usepackage{xcolor}
\usepackage{hyperref}
\usepackage[utf8]{inputenc}
\usepackage[russian]{babel}

\definecolor{linkcolor}{HTML}{799B03} % цвет ссылок
\definecolor{urlcolor}{HTML}{799B03} % цвет гиперссылок
 
\hypersetup{pdfstartview=FitH,  linkcolor=linkcolor,urlcolor=urlcolor, colorlinks=true}
\pagestyle{empty}

%opening
%\title{\Huge Резюме}
%\author{\Huge Тюльбашев Владислав Сергеевич}

\begin{document}
%\maketitle
\begin{center}
	\hugeТюльбашев Владислав Сергеевич\\
	\small<vtyulb@vtyulb.ru>
\end{center}
\section{Последние проекты} {
	\subsection{BSA-Analytics, июль 2014 - по текущий момент} {
		\href{https://github.com/vtyulb/BSA-Analytics}{https://github.com/vtyulb/BSA-Analytics} \\
		BSA-Analytics - программа для обработки данных с радиотелескопа BSA. Основные фичи: инсталлятор, автообновления, пользовательская документация,
		присутствие GUI интерфейса, не отменяющее возможность работать в голой консоли. Qt, дополнительные библиотеки не использовались.
		
		Возможности программы: просмотр данных с BSA, прямой поиск пульсаров, поиск через Фурье-обработку, поиск коротких транзиентов,
		построение динамических спектров найденных пульсаров и кандидатов в транзиенты, конвертация дорожек с данными в звук, оценка плотности
		потока найденных объектов, нормализация данных путем поиска специальных меток в данных, возможность полного переформатирования
		данных для более удобной работы (особенности телескопа). 
		
		BSA-Analytics используется в Пущинской радиоастрономической обсерватории для решения комплекса задач. 
		С помощью программы были найдено множество объектов, по результатам работ\\ опубликовано 3 статьи, 2 в печати, 2 готовятся к печати. Примеры статей:\newline
		\href{http://adsabs.harvard.edu/abs/2016ARep...60..220T}{Поиск пульсаров на 111МГц} \newline
		\href{http://comet.sai.msu.ru/~gmr/AC/AC1624.pdf}{Открытие новых пульсаров на радиотелескопе БСА ФИАН I} \newline
		\href{http://comet.sai.msu.ru/~gmr/AC/AC1625.pdf}{Открытие новых пульсаров на радиотелескопе БСА ФИАН II} \newline
	}
	
	\subsection{QLiveBittorrent, лето 2013} {
		\href{https://github.com/vtyulb/BSA-Analytics}{https://github.com/vtyulb/QLiveBittorrent}\\
		Битторрент клиент с возможностью скачивания и просмотра файлов с произвольного места одновременно. Linux-only, Qt, FUSE, libtorrent.\\
		Подробнее в \href{http://habrahabr.ru/post/185770/}{статье на хабре}
	}
	
	\subsection{Labyrus, октябрь 2012 - май 2013} {
		\href{https://github.com/vtyulb/Labyrus}{https://github.com/vtyulb/Labyrus}\\
		Кроссплатформенная сетевая 3D игрушка. Qt, OpenGL, движок полностью свой.\\
		Подробнее в \href{http://habrahabr.ru/post/177807}{статье на хабре}
	}
}

\section{Технологии} {
	Большой опыт: Qt, C++\\
	Поверхностный опыт: Си, Ассемблер (интел), OpenGL, FUSE, libtorrent, OpenCV, bash\\
}

\section{Образование} {
	2011 - 2013: СУНЦ МГУ, 10-11 класс\\
	2013 - 2016: ВМК МГУ
}

\section{Общая информация} {
	Бывший олимпиадник, призер всероссийской олимпиады школьников по программированию 2012 и 2013. Один из организаторов
	Пущинской новогодней олимпиады (написание около половины задач, настройка сервера, помощь на локальной площадке).
	Последние 4 года держу свой сервер для поддержки ряда сервисов. Предпочитаю писать на Qt/C++, живу под ArchLinux.
}

\section{Ожидание по работе} {
	Data Scientist / Программист-исследователь.
	Ожидаемая заработная плата: 100-110 т.р. в месяц.
}

%\section{Технологии} {
%    \subsection{Pascal}{}
%    \subsection{C, C++}{}
%    \subsection{latex}
%    \subsection{brainfuck}
%    \subsection{Qt}{}{}
%    \subsection{OpenGl}{}
%    \subsection{FUSE}{}
%    \subsection{libtorrent-rasterbar}{}
%    \subsection{Навыки установки и администрирования операционных систем семейств Windows, Linux, в особенности ArchLinux.}{}
%    \subsection{VirtualBox}{}
%    \subsection{Алгоритмы}{
%	\subsubsection{Динамическое программирование}{}
%	\subsubsection{Графы}{Обход в ширину, поиск в глубину, Дейкстра, Флойд, Форд-Беллман,
%	    \paragraph{Каркасы}{Алгоритм краскала, Алгоритм Прима}
%	    \paragraph{Потоки}{Форд-Фалкерсон, Диница, масштабирование}
%	}
%	\subsubsection{Структуры данных}{
%	    \paragraph{Дерево отрезков}{}
%	    \paragraph{Дерево фенвика(в том числе многомерное)}{}
%	    \paragraph{Декартого дерево}{}
%	    \paragraph{Сплей-дерево}{}
%	    \paragraph{Куча, очередь, стэк, дек}{}
%	    \paragraph{Персистентные}{Все то же самое, но персистентное, за исключением сплей-дерева, дерева фенвика, кучи.}
%	}
%	\subsubsection{Прочие сложные и не очень алгоритмы}{sqrt-декомпозиция, разбор~выражений, длинная~арифметика, 
%			суффиксный~массив, Z-функция, префикс-функция, поиск~мостов~и~точек~сочленения, теория~игр, 
%			вычислительная~геометрия, LCA}
%    }
%}


\end{document}
