\documentclass[a4paper, 8pt]{article}
\usepackage[a4paper, left=20mm, right=10mm, top=20mm, bottom=20mm]{geometry}
\usepackage{xcolor}
\usepackage{hyperref}
\usepackage[utf8]{inputenc}
\usepackage[russian]{babel}
\definecolor{linkcolor}{HTML}{799B03} 

%opening
\title{\Huge Резюме}
\author{\Huge Тюльбашев Владислав Сергеевич}

\begin{document}
\maketitle
\section{Олимпиадные достижения} {
	Призер всероссийской олимпиады школьников по программированию 2012(23 место) \newline
	Призер всероссийской олимпиады школьников по программированию 2013(22 место)
}

\section{Последний проект} {
	\href{https://blog.vtyulb.ru/projects/BSA-Analytics}{BSA-Analytics} - программа для обработки данных с 
	Пущинского телескопа BSA. С помощью программы были найдено 9 пульсаров, опубликованы 3 статьи.
	\subsection{Астрономический журнал} {\href{http://adsabs.harvard.edu/abs/2016ARep...60..220T}{Поиск пульсаров на 111МГц} }
	\subsection{Астрономический циркуляр} {
		\href{http://comet.sai.msu.ru/~gmr/AC/AC1624.pdf}{Открытие новых пульсаров на радиотелескопе БСА ФИАН I} \newline
		\href{http://comet.sai.msu.ru/~gmr/AC/AC1625.pdf}{Открытие новых пульсаров на радиотелескопе БСА ФИАН II}
	}
}

\section{Публикации на хабре} {
    \href{http://habrahabr.ru/post/185770/}{QLiveBittorrent} \newline
    \href{http://habrahabr.ru/post/177807/}{Labyrus}
}

\section{Образование} {
    Общее среднее (полное) - СУНЦ МГУ.
    Сейчас учусь на ВМК (4 курс).
}

%\section{Технологии} {
%    \subsection{Pascal}{}
%    \subsection{C, C++}{}
%    \subsection{latex}
%    \subsection{brainfuck}
%    \subsection{Qt}{}{}
%    \subsection{OpenGl}{}
%    \subsection{FUSE}{}
%    \subsection{libtorrent-rasterbar}{}
%    \subsection{Навыки установки и администрирования операционных систем семейств Windows, Linux, в особенности ArchLinux.}{}
%    \subsection{VirtualBox}{}
%    \subsection{Алгоритмы}{
%	\subsubsection{Динамическое программирование}{}
%	\subsubsection{Графы}{Обход в ширину, поиск в глубину, Дейкстра, Флойд, Форд-Беллман,
%	    \paragraph{Каркасы}{Алгоритм краскала, Алгоритм Прима}
%	    \paragraph{Потоки}{Форд-Фалкерсон, Диница, масштабирование}
%	}
%	\subsubsection{Структуры данных}{
%	    \paragraph{Дерево отрезков}{}
%	    \paragraph{Дерево фенвика(в том числе многомерное)}{}
%	    \paragraph{Декартого дерево}{}
%	    \paragraph{Сплей-дерево}{}
%	    \paragraph{Куча, очередь, стэк, дек}{}
%	    \paragraph{Персистентные}{Все то же самое, но персистентное, за исключением сплей-дерева, дерева фенвика, кучи.}
%	}
%	\subsubsection{Прочие сложные и не очень алгоритмы}{sqrt-декомпозиция, разбор~выражений, длинная~арифметика, 
%			суффиксный~массив, Z-функция, префикс-функция, поиск~мостов~и~точек~сочленения, теория~игр, 
%			вычислительная~геометрия, LCA}
%    }
%}

\section{Контакты}{
    \href{http://habrahabr.ru/users/vtyulb/}{habrahabr} \newline
    \href{https://github.com/vtyulb}{github} \newline
    \textbf{<vtyulb@vtyulb.ru>}
}

\end{document}
